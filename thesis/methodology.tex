\section{Methodology}

\subsection{Description of the data}
% Data verzameling en beschrijving van de data
% Hoe is de data verzameld, en hoe heb jij die data verkregen?
% Wat staat er in de data? Niet alleen maar een technisch verhaal, maar ook inhoudelijk. DE lezer moet een goed idee krijgen over de technische inhoud en wat het betekent.
For classifying fake news, Wang's Liar dataset will be used \cite{wang2018}. 
The Liar dataset contains 12.791 short statements from Politifact.com, which are labeled manually by a Politifact.com editor on truthfulness. 
The statements are an average of 18 tokens long, and the topics vary from different political subjects, as can be seen in figure 1\todo{Check whether this number is still correct}.
Truthfulness is evaluated by assigning one of 6 labels, ranging from \textit{pants-on-fire} to \textit{true}. 
The distribution of statements across labels can be seen in figure 3\todo{Check whether this number is still correct}.

\begin{figure}[h]
    \centering
    \includegraphics[scale=0.25]{subjectwordcloud}
    \caption{An overview of all statement topics in the Liar dataset.}
\end{figure}

For each statement, the dataset contains an id, a label, a subject, a speaker, the function title of the speaker, the affiliated state and political affiliation, the context of the statement and a vector with a truthfulness history.
Wang introduced this truthfulness history to boost the prediction scores, as speakers with a track record of lying are expected to have a lower chance of speaking the truth when classifying new statements.
However, for our application we are only interested in the statement itself and its corresponding label. 
Due to cheapness and spreadability, a large amount of fake news is spread over social media \cite{shu2017}. 
This means author information and metadata will not readily be available in real world circumstances.

The original dataset has been split beforehand into a test, train and validation set. 
The train set contains 80\% of the total amount of statements, while the test and validation set both contain approximately 10\% of the statements. 

\begin{figure}[h]
    \centering
    \includegraphics[scale=0.5]{folddistribution}
    \caption{Distribution of the total dataset.}
\end{figure}

\subsection{Data preprocessing and cleaning}
\subsubsection{Filtering statements}
The original dataset contained statements ranging from 1 sentence to 19. 
On closer inspection of statements with the high amounts of sentences, it was found that not all statements were processed from source files into dataframes correctly.
As a result, records of some different statements were joined together, forming a single string.
To combat this, the following regular expression was used to filter those statements out:\\
\\
\verb/\.json\\t(mostly-true|true|half-true|false|barely-true|pants-fire)\/\\
\\
After applying this regular expression, the total amount of sentences in the statements were reduced from a maximum of 19 to a maximum of 11. 

\subsubsection{Reducing labels}

Wang's main objective was to classify fake news into a fine-grained category of fakeness. 
His best model obtained a 27\% accuracy on the validation set \cite{wang2018}.
For our main research question, we only aim to predict whether the statements are fake news or not. 
Because of this, for the rest of this research, the prediction labels from the original dataset will be reduced from 6 labels to 3 labels, according to the recoding schema used by Khurana \cite{khurana2017}.
The 6 labels are reduced to the following 3 labels: \texttt{true}, \texttt{half-true} and \texttt{false}.

\begin{figure}[h]
    \centering
    \includegraphics[scale=0.5]{labeldistribution}
    \caption{Distribution of the statements over the labels, before and after reducing the labels from 6 to 3.}
\end{figure}


\subsection{Methods}
% Hoe je je vraag gaat beantwoorden.
% Dit is de langste sectie van je scriptie. 
% Als iets erg technisch wordt kan je een deel naar de Appendix verplaatsen. 
% Probeer er een lopend verhaal van te maken.
% Het is heel handig dit ook weer op te delen nav je deelvragen:

\subsubsection{RQ1}

\subsubsection{RQ2}

\subsubsection{RQ3}
