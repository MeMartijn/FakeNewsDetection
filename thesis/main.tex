\documentclass[a4paper,pdf]{article} % gebruik acm style voor je scriptie: [format=acmsmall, screen=true, review=false]{acmart} 


%\documentclass[sigconf]{acmart} 
%\documentclass[sigconf,format=acmsmall, screen=true, review=false]{acmart} 
\usepackage{amsmath}
\usepackage{amsfonts}
\usepackage{amssymb}
\usepackage{hyperref}
\usepackage{pdfpages} % http://mirror.unl.edu/ctan/macros/latex/contrib/pdfpages/pdfpages.pdf
\usepackage{booktabs} 


\usepackage[utf8]{inputenc}
\usepackage{graphicx}
\graphicspath{ {./images/} }
\usepackage[colorinlistoftodos]{todonotes} % handig voor commentaar: gebruik \todo{}, zie ftp://ftp.fu-berlin.de/tex/CTAN/macros/latex/contrib/todonotes/todonotes.pdf
\usepackage{listings}
 
\usepackage{tcolorbox}
\usepackage{float}
\usepackage{caption}
\usepackage{subcaption}

% when writing in Dutch
%\usepackage[dutch]{babel}
%\selectlanguage{dutch}


% linenumbering  See https://texblog.org/2012/02/08/adding-line-numbers-to-documents/
\usepackage{lineno}
\linenumbers

\newcommand{\shorttitle}{} % Put your short title here
\begin{document}

\begin{center}

\vspace{2.5cm}

% [CHANGE] The title of your thesis. If your thesis has a subtitle, then this
% should appear right below the main title, in a smaller font.
\begin{Huge}
Fake news: an algorithmic perspective on fact-checking
\end{Huge}

\vspace{1.5cm}

% [CHANGE] Your full name. In case of multiple names, you can include their
% initials as well, e.g. "Jan G.J. van der Wegge".
Martijn B.J. Schouten\\
% [CHANGE] Your student ID, as this has been assigned to you by the UvA
% administration.
11295562

\vspace{1.5cm}

% [DO NOT CHANGE]
Bachelor thesis\\
% Whether your Bachelor thesis is 6 ECTS (regular) or 9 ECTS (Honours
% programme).
Credits: 12 EC

\vspace{0.5cm}

% [DO NOT CHANGE] The name of the educational programme.
Bachelor's degree Information Science

\vspace{0.25cm}

% [DO NOT CHANGE] The addess of the educational programme.
University of Amsterdam\\
Faculty of Science\\
Science Park 904\\
1098 XH Amsterdam

\vspace{4cm}

\emph{Supervisor}\\
% The name of your supervisor. Include the titles of your supervisor,
% as well as the initials for *all* of his/her first names.
Dr. M. J. Marx

\vspace{0.25cm}

% The address of the institute at which your supervisor is working.
% Be sure to include (1) institute (is appropriate), (2) faculty (if
% appropriate), (3) organisation name, (4) organisation address (2 lines).
ILPS, IvI\\
Faculty of Science\\
University of Amsterdam\\
Science Park 904\\
1098 XH Amsterdam

\vspace{1.5cm}

% The date at which you will finalize and submit your thesis.
2019-06\todo{Add a final date}

\end{center}


\pagebreak

%\todototoc
%\listoftodos

\pagebreak

\begin{abstract}
% [CHANGE] 
\todo{Add an abstract}
\end{abstract}

\pagebreak

\tableofcontents

\pagebreak

% Here you input all your sections in seperate files

\section{Introduction}
% Bevat je onderzoeksvraag (of vragen)
% Plaatst je vraag in de bestaande literatuur.
% Je onderzoeksvraag is leidend voor je hele scriptie. Alles wat je doet moet uiteindelijk terug te voeren zijn op 1 doel: het beantwoorden van die vraag. 
The ability to broadcast information on a large scale has been in the hands of large publishing organizations in the pre-Internet era, but nowadays everyone can share news via social media \cite{howell2013}. 
This introduces risks on validity and authenticity of news, as social media and digital platforms can speed up the spread of falsehoods without much effort from the author \cite{europeancommission2018}. 

As a matter of fact, 63\% of adults in the United States prefer to read their news on the Internet. 
Young adults take the lead: 76\% of adults between the ages 18 and 49 get their primary news consumption via the web, compared to just 43\% for adults of 50 years and older \cite{mitchell2018}.
As time passes by, social media is slowly becoming the primary source of news for more and more people. 

The main danger of this development is that human perception is often skewed with regards to objectivity of facts. 
Naïve realism let consumers of news belief that their perception is right, while other's perceptions are uninformed. 
Furthermore, confirmation bias results in consumers preferring information that confirms beliefs they already have \cite{shu2017}. 
This makes consumers vulnerable for the spread of misinformation or fake news. 

According to the European Commission, \textit{"disinformation - or fake news - consists of verifiably false or misleading information that is created, presented and disseminated for economic gain or to intentionally deceive the public, and may cause public harm"} \cite{europeancommission2018}. 
The answer to the problem of fake news as of recently has been to manually fact-check statements on validity, but, as Shu et al. underlines, one of the downsides to this approach is that fake news typically relates to newly emerging, time-critical events. 
This means the real news may not be fully verified by proper knowledge bases due to a lack of contradicting claims \cite{shu2017}. 
An automated approach would both help in solving the problem of human subjectivity and the speed at which false information is spread in the current news spreading landscape. 

Natural language processing has been in rapid development over the past years. 
With the releases of OpenAI's GPT-2 model in February of this year and Google's BERT in the autumn of 2018, state-of-the-art pre-trained textual embedding techniques have shown promising results on various classification tasks \cite{radford2019}\cite{devlin2018}. 
Although fake news classification has been attempted before \cite{wang2018}\cite{khurana2017}, performance has been rather low. However, these new pre-trained textual embeddings have not yet been used in the fight against disinformation. 

This thesis is focussed on the following research question: what is the performance of combinations of pre-trained embedding techniques with machine learning algorithms when classifying fake news?
This main question will be answered through the results of the following subquestions:

\begin{description}
\item[RQ1]
\item[RQ2]
\item[RQ3]
\end{description}

% Je Evaluatie sectie bevat evenveel subsecties als je deelvragen hebt. En in elke sectie beantwoord je dan die deelvraag met behulp van de vragen op het onderste niveau.
% In je conclusies kan je dan je hoofdvraag gaan beantwoorden op basis van al het eerder vergaarde bewijs.


\paragraph{Overview of thesis}
% Hier geef je even kort weer wat in elke sectie staat.
\section{Related Work}

\subsection{RQ1}
Fake news as a term only caught public attention starting from the end of 2016, during the Presidential Elections of the United States \cite{googletrends2019}.   

\subsection{RQ2}
In the last couple of years, using transfer learning for natural language processing has given promisable results. The following sentence embeddings will be used to detect fake news:

\begin{itemize}
    \item Bag of Words as a baseline for performance of non-pretrained embeddings;
    \item Facebook's InferSent \cite{conneau2017};
    \item ELMo from the Allen Institute for Artificial Intelligence \cite{peters2018};
    \item OpenAI's GPT-2 \cite{radford2019};
    \item Transformer-XL \cite{dai2019};
    \item Microsoft's MT-DNN  \cite{liu2019};
    \item and Google's BERT \cite{devlin2018}.
\end{itemize}

\subsection{RQ3}
Aligned with the original research on this dataset by Wang \cite{wang2018}, the following machine learning algorithms will be used to test the applicability of the abovementioned embedding techniques: 
\begin{itemize}
    \item SVMs;
    \item Logistic regression;
    \item Bi-LSTMs;
    \item CNNs.
\end{itemize}
\section{Methodology}

\subsection{Description of the data}
% Data verzameling en beschrijving van de data
% Hoe is de data verzameld, en hoe heb jij die data verkregen?
% Wat staat er in de data? Niet alleen maar een technisch verhaal, maar ook inhoudelijk. DE lezer moet een goed idee krijgen over de technische inhoud en wat het betekent.
For classifying fake news, Wang's Liar dataset will be used \cite{wang2018}. 
The Liar dataset contains 12.791 short statements from Politifact.com, which are labeled manually by a Politifact.com editor on truthfulness. 
The statements are an average of 18 tokens long, and the topics vary from different political subjects, as can be seen in figure 1\todo{Check whether this number is still correct}.
Truthfulness is evaluated by assigning one of 6 labels, ranging from \textit{pants-on-fire} to \textit{true}. 
The distribution of statements across the original 6 labels can be seen in table 2\todo{Check whether this number is still correct}.

\begin{figure}[h]
    \centering
    \includegraphics[scale=0.25]{subjectwordcloud}
    \caption{An overview of all statement topics in the Liar dataset.}
\end{figure}

For each statement, the dataset contains an id, a label, a subject, a speaker, the function title of the speaker, the affiliated state and political affiliation, the context of the statement and a vector with a truthfulness history.
An example of such a data entry can be seen in table 1.\todo{Check whether this number is still correct}
Wang introduced this truthfulness history to boost the prediction scores, as speakers with a track record of lying are expected to have a lower chance of speaking the truth when classifying new statements.
However, for our application we are only interested in the statement itself and its corresponding label. 
Due to cheapness and spreadability, a large amount of fake news is spread over social media \cite{shu2017}. 
This means author information and metadata will not readily be available in real world circumstances.

\begin{table}[]
    \centering
    \begin{tabular}{ll}
        \hline
        id                     & 11044.json                                                      \\ \hline
        label                  & pants-fire                                                      \\ \hline
        statement              & The Mexican government forces many bad people into our country. \\ \hline
        subjects               & foreign-policy,immigration                                      \\ \hline
        speaker                & donald-trump                                                    \\ \hline
        speaker\_job           & President-Elect                                                 \\ \hline
        state                  & New York                                                        \\ \hline
        party                  & republican                                                      \\ \hline
        context                & an interview with NBC's Katy Tur                                \\ \hline
        mostly\_true\_count    & 37                                                              \\ \hline
        half\_true\_count      & 51                                                              \\ \hline
        barely\_true\_count    & 63                                                              \\ \hline
        false\_count           & 114                                                             \\ \hline
        pants\_on\_fire\_count & 61                                                              \\ \hline
    \end{tabular}
    \caption{An example entry in the Liar dataset.}
\end{table}

The original dataset has been split beforehand into a test, train and validation set. 
The train set contains 80\% of the total amount of statements, while the test and validation set both contain approximately 10\% of the statements. 

\subsection{Data preprocessing and cleaning}
\subsubsection{Filtering statements}
The original dataset contained statements ranging from 1 sentence to 19. 
On closer inspection of statements with the high amounts of sentences, it was found that not all statements were processed from source files into dataframes correctly.
As a result, records of some different statements were joined together, forming a single string.
To combat this, the following regular expression was used to filter those statements out:\\
\\
\verb/\\.json\\t(mostly-true|true|half-true|false|barely-true|pants-fire)\/\\
\\
After applying this regular expression, the total amount of sentences in the statements were reduced from a maximum of 19 to a maximum of 11. 

\subsubsection{Reducing labels}

Wang's main objective was to classify fake news into a fine-grained category of fakeness \cite{wang2018}.
For our main research question, we aim to predict whether the statements are fake news or not. 
This means the statements do not necessarily need to be distinguished into these fine-grained categories.
Because of this, the classifiers used to predict fake news in this research will be trained on the original 6 labels, Khurana's division into three labels \cite{khurana2017}, and a binary classification.
The division from the original 6 labels into the lesser amounts of labels can be seen in table 2. 
This way, we can better compare performance of pre-trained embeddings to existing research on this dataset.

\begin{table}[]
    \centering
    \begin{tabular}{|l|l|l|}
        \hline
        \textbf{6 labels}                                              & \textbf{3 labels}                                                          & \textbf{2 labels}                                                          \\ \hline
        \begin{tabular}[c]{@{}l@{}}true\\ (16.1\%)\end{tabular}        & \multirow{2}{*}{\begin{tabular}[c]{@{}l@{}}true\\ (35,3\%)\end{tabular}}   & \multirow{3}{*}{\begin{tabular}[c]{@{}l@{}}true\\ (55,8\%)\end{tabular}}   \\ \cline{1-1}
        \begin{tabular}[c]{@{}l@{}}mostly-true\\ (19.2\%)\end{tabular} &                                                                            &                                                                            \\ \cline{1-2}
        \begin{tabular}[c]{@{}l@{}}half-true\\ (20.5\%)\end{tabular}   & \begin{tabular}[c]{@{}l@{}}half-true\\ (20.5\%)\end{tabular}               &                                                                            \\ \hline
        \begin{tabular}[c]{@{}l@{}}barely-true\\ (16.4\%)\end{tabular} & \multirow{3}{*}{\begin{tabular}[c]{@{}l@{}}false\\ (44,19\%)\end{tabular}} & \multirow{3}{*}{\begin{tabular}[c]{@{}l@{}}false\\ (44,19\%)\end{tabular}} \\ \cline{1-1}
        \begin{tabular}[c]{@{}l@{}}false\\ (19.6\%)\end{tabular}       &                                                                            &                                                                            \\ \cline{1-1}
        \begin{tabular}[c]{@{}l@{}}pants-fire\\ (8.19\%)\end{tabular}  &                                                                            &                                                                            \\ \hline
    \end{tabular}
    \caption{Distribution of labels from the original label distribution when reducing the amount of labels.}
\end{table}

\subsection{Methods}
% Hoe je je vraag gaat beantwoorden.
% Dit is de langste sectie van je scriptie. 
% Als iets erg technisch wordt kan je een deel naar de Appendix verplaatsen. 
% Probeer er een lopend verhaal van te maken.
% Het is heel handig dit ook weer op te delen nav je deelvragen:

\subsubsection{Applying embedding techniques}
As our main research question is focussed on pre-trained word embeddings, the first step in the classification process is to turn the statements of the Liar dataset into vectors. 
For this purpose, the Flair framework will be used. 
Flair contains interfaces for turning words into embeddings, built on the PyTorch platform \cite{flairrepo}\cite{pytorch}. 
Using Flair, we have access to the following 5 state-of-the-art pre-trained embedding techniques: 
\begin{itemize}
    \item ELMo (Embeddings from Language Models) \cite{peters2018};
    \item BERT \cite{devlin2018};
    \item Generative Pre-Training (GPT) \cite{radford2018};
    \item Transformer-XL \cite{dai2019};
    \item Flair \cite{akbik2019}.
\end{itemize}

To apply these embeddings, the Flair framework first requires a sentence object to be created \cite{flairsentence}.
For dividing the statements into sentences, the \texttt{sent\_tokenize} function from the \texttt{nltk} package will be used \cite{nltktokenize}. 
After applying this split, Flair's sentence object takes care of tokenization and applying the selected embedding technique.

\paragraph{Embeddings from Language Models (ELMo)}
The first word embedding technique, ELMo, is based on a bidirectional language model. 
ELMo embeddings are different from regular word embeddings, because each token is a function of the entire input sentence.
The underlying neural network architecture consists of a bidirectional LSTM trained on a large corpus.
This corpus contained approximately 5.5 billion tokens, crawled from Wikipedia and WMT 2008-2012.

The representation is formed from combining internal states of the LSTM. 
The higher level LSTM states capture context-dependent aspects of word meaning, while the lower level states model aspect of syntax.
Combining these internal states allows for rich representations that capture both complex characteristics of word use (syntax and semantics) and how word uses vary across linguistic contexts\cite{peters2018}.

The used model for our use will be the original ELMo model, containing 93.6 million parameters.
Each created word vector has a length of 3072. 

\paragraph{Generative Pre-Training (GPT)}
OpenAI's GPT model is among the first text embedding methods to use the Transformer architecture by Vaswani et al., as described in section 2.2 \cite{vaswani2017}.
Radford et al. chose specifically for this architecture, as these types of models allow for a more structured memory for handling long-term dependencies in text.
This results in a robust transfer performance across different tasks. 

GPT's training procedure consists of two stages: learning a language model on a large corpus of text, and a fine-tuning stage, where the model is adapted to a task with labeled data.
For learning the language model, the BooksCorpus dataset was used, which contains over 7000 books, and is of comparable size to the datasets used to pre-train ELMo embeddings \cite{radford2018}.

Each created word vector has a length of 1536.

\paragraph{Bidirectional Encoder Representations from Transformer (BERT)}
This embedding technique is also based on the Transformer architecture described in section 2.2, but differs from the GPT Transformer, as it is trained using bi-directional self-attention instead of GPT's constrained left-only self-attention.
Pre-training is conducted on 2 unsupervised learning tasks. 
The first utilizes a masked language model (MLM), which randomly masks some of the tokens from the input with the objective to predict the masked word, based only on its context.
This MLM fuses the context to the left and to the right, allowing pre-training of a deep bi-directional Transformer architecture.
In the second task, the model takes a sentence, and predicts the next sentence in the sequence. 
The first task is aimed on word level, while this second learning task aims at the model's sentence level understanding \cite{devlin2018}. 

The model used for embedding our statements will be the original BERT Uncased model, which contains 110 million parameters and 12 layers. 
Pre-training for this model was conducted using a combination of the BooksCorpus (800 million words) and Wikipedia (2.500 million words).
Each created word vector has a length of 3072. 

\paragraph{Transformer-XL}

\paragraph{Flair}

\subsubsection{RQ1}

\subsubsection{RQ2}

\subsubsection{RQ3}

\section{Evaluation}
% Met een subsectie voor elke deelvraag.
% In hoeverre is je vraag beantwoord?
% Een mooie graphic/visualisatie is hier heel gewenst.
% Hou het kort maar krachtig.

\subsection{RQ1}

\subsection{RQ2}

\subsection{RQ3}
\section{Conclusions}
The main aim of this paper has been to apply new techniques to battle the problem of fake news.
As social media emerges as the main platform for consumption of news, the spread of falsehoods is increasing. 
An automated way to detect lies could assist to combat this development by both supplying tools for factchecking organizations to speed up their process, or to counter subjective human intuition.

Various state-of-the-art word embedding techniques have been used in combination with a set of both neural and non-neural classification algorithms to classify fake news.
To turn raw textual data into a uniform shape, both padding and pooling techniques have been compared. 
The amount of padding and the robust performance even on lower amounts of words allowed us to believe that pre-trained word embeddings above all contain contextual data, making an increase of word vectors unnecessary, as a small amount of word vectors already contain a large chunk of data concerning the context of the whole sentence. 

To conclude, the highest achieved accuracy was 52,96\% with a combination of BERT, which is a Transformer-based embedding technique, and a logistic regression.
This combination performed almost 4\% better than recent research to this same dataset, but using traditional linguistic features. 

% your refs

\pagebreak

\bibliographystyle{plain}
\bibliography{bibliography}

\appendix

%\input{appendix}
 
\end{document}
